\documentclass{report}

\usepackage[usenames,dvipsnames]{color}
\usepackage{xcolor}
\usepackage{listings}
\usepackage{caption}
\usepackage{courier}
\usepackage{fancyvrb}
\usepackage{xparse}
\usepackage{fullpage}
\usepackage{colortbl}
\DefineShortVerb{\|}
\DeclareCaptionFont{white}{\color{white}}
\DeclareCaptionFormat{listing}{\colorbox{gray}{\parbox{\textwidth}{#1#2#3}}}
\captionsetup[lstlisting]{format=listing,labelfont=white,textfont=white}
\lstset{numbers=left, language=Python, numbersep=5pt, numberstyle=\tiny, basicstyle=\ttfamily}
\lstdefinestyle{numbers}{numberstyle=\tiny}
\renewcommand*\arraystretch{2.0}
\NewDocumentCommand \method{ m m g g g g g g g }{%
	{\tt{\textcolor{red}{#1} #2(}%
		\IfValueTF{#3}{\textcolor{blue}{#3}%
			\IfValueTF{#4}{\textbf{,}\textcolor{blue}{#4}%
				\IfValueTF{#5}{\textbf{,}\textcolor{blue}{#5}%
					\IfValueTF{#6}{\textbf{,}\textcolor{blue}{#6}%
						\IfValueTF{#7}{\textbf{,}\textcolor{blue}{#7}%
						}{\tt{)}}%
					}{\tt{)}}%
				}{\tt{)}}%
			}{\tt{)}}%
		}{\tt{)}}%
	}%
}

\definecolor{lightblue}{rgb}{0.8,0.8,0.8}
\definecolor{lightyellow}{rgb}{0.9,0.9,0.9}

\title{SpikeDB User Manual}
\author{Brandon Aubie}
\date{Last Updated: March 1 2011}

\begin{document}
\newcolumntype{b}{>{\columncolor{lightblue}}c}
\newcolumntype{y}{>{\columncolor{lightyellow}}l}

\maketitle
\tableofcontents 

\chapter{Introduction}
SpikeDB is a database and analysis tool for electrophysiological recordings done with Spike. It runs on Linux and Mac (and probably Windows).

\chapter{Basic User Interface}
\section{Browse Files}
\subsection{Plots}
\subsubsection{General Usage}
\begin{itemize}
	\item \textbf{Zoom} - Left click and drag horizontally to zoom in on a subsection of data.  
	\item \textbf{Export Data} - Right click anywhere on the plot to bring up the options menu and select |Export Data|. This allows you to export the plotted data in CSV files that are ready for import into Excel or for use in other graphing software such as GLE.
\end{itemize}

\subsubsection{Spike Raster}
The spike raster is a built in plot that displays the stimuli as red (channel 1) and blue (channel 2) lines and spikes as black dots. Zooming in on this plot will limit the spike times available to the quick analysis plot on the right.

\subsubsection{Quick Analysis}
By default, this plot will display the mean number of spikes per trial. Other analysis plugins are available in the drop down box or additional plugins can be loaded by clicking the |Open| icon. Generally, it is wise to use plugins that operate on selected files only here as no text display is available. For more general analysis on many files use the Analysis tab. That said, if multiple files of the same type are selected, the plots can be overlaid for easy comparison.


\chapter{Analysis Plug-In Module}

\section{Basic Usage}
The Analysis Plug-In Module allows you to use the Python scripting language write custom analysis routines on one or many Spike recording files. Each Python script will have the |SpikeDB| object available to it.  This object provides access to all of the data held within SpikeDB as well as a host of methods useful for analysis.

\section{Reference}
All functions listed below are accessed via the SpikeDB object.  For example, |SpikeDB.getFiles(True)| calls the |getFiles| method.

\subsection{\method{void}{filterSpikesAbs}{float minSpikeTime}{float maxSpikeTime}}
\subsubsection{Parameters}
\begin{center}
\begin{tabular}{|b|y|}
	\hline
	|minSpikeTime| & Minimum absolute spike time.\\
	\hline
	|maxSpikeTime| & Maximum absolute spike time.\\
	\hline
\end{tabular}
\end{center}
\subsubsection{Description}
Filters the spikes for every file based on the absolute time of the spike in the file.

\subsection{\method{void}{filterSpikesRel}{float minSpikeTime}{float maxSpikeTime}}
\subsubsection{Parameters}
\begin{center}
\begin{tabular}{|b|y|}
	\hline
	|minSpikeTime| & Minimum relative spike time.\\
	\hline
	|maxSpikeTime| & Maximum relative spike time.\\
	\hline
\end{tabular}
\end{center}
\subsubsection{Description}
Filters the spikes for every file based on the time of the spike relative to the stimuli onset and offsets. A spike is included only if it falls within a stimulus onset+|minSpikeTime| and stimulus offset+|maxSpikeTime|. When both channel 1 and channel 2 are active a spike is included if it falls within the relative timing of either stimulus. To include spikes prior to stimulus onset set |minSpikeTime| to a value less than zero.


\subsection{\method{list}{getCells}}
\subsection{\method{list}{getFiles}{bool onlySelected}}
\subsection{\method{float}{mean}{list values}}
\subsection{\method{void}{plotClear}}
\subsection{\method{void}{plotLine}{list xValues}{list yValues}{list errValues}}
\subsection{\method{void}{plotSetLineWidth}{float lineWidth}}
\subsection{\method{void}{plotSetPointSize}{float pointSize}}
\subsection{\method{void}{plotSetRGBA}{float red}{float green}{float blue}{float alpha}}
\subsection{\method{void}{plotXLabel}{string XLabel}}
\subsection{\method{void}{plotXMin}{float minXValue}}
\subsection{\method{void}{plotXMax}{float maxXValue}}
\subsection{\method{void}{plotYLabel}{string YLabel}}
\subsection{\method{void}{plotYMin}{float minYValue}}
\subsection{\method{void}{plotYMax}{float maxYValue}}
\subsection{\method{void}{plotYMax}{float maxYValue}}
\subsection{\method{float}{stddev}{list values}}
\subsection{\method{void}{write}{string text}}
\subsubsection{Parameters}
\begin{center}
\begin{tabular}{|b|y|}
	\hline
	|text| & Text to print to SpikeDB output window.\\
	\hline
\end{tabular}
\end{center}
\subsubsection{Description}
This function is used internally to print text to the SpikeDB output window and is generally not needed by analysis script writers. Standard Python output functions like |print| work just fine and print to the SpikeDB output window as expected. Errors are also printed to the SpikeDB window. Note that the output window is only available in the Analysis tab and not in the quick analysis plot.

\section{Examples}
\subsection{Mean Spike Times}
\begin{lstlisting}[label=codeMean,caption=Calculating the mean spike counts.]
files = SpikeDB.getFiles(True)
for f in files:
	means = []
	err = []
	x = []
	for t in f['trials']:
		count = []
		x.append(t['xvalue'])	
		for p in t['spikes']:
			count.append(len(p))
		means.append(SpikeDB.mean(count))
		err.append(SpikeDB.stddev(count))
	SpikeDB.plotXLabel(f['xvar'])
	SpikeDB.plotYLabel('Mean Spike Count')
	SpikeDB.plotLine(x,means,err)
\end{lstlisting}

\end{document}
